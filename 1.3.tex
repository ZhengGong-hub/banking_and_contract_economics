Julius Baer's lending relationship with Signa Group shows a mixed approach that didn't quite work out - it wasn't really doing either relationship banking or arm's-length finance properly. \textbf{Relationship banking} is about personal connections and soft information, while \textbf{arm's-length finance} is more about standard contracts with hard information and easy-to-sell collateral.

Some parts of what Julius Baer did looked like \textbf{relationship banking}. Signa's complicated structure needed special relationship knowledge, and that big CHF 144 million write-down shows they were in it for the long haul. When Signa started having problems, Julius Baer tried to work things out instead of just pulling the plug, which is typical relationship banking behavior.

But Julius Baer also had some \textbf{arm's-length finance} habits, especially with how they handled collateral. They used \textbf{illiquid share pledges} and didn't really check up on them regularly - they just assumed the initial deal was good enough. Because they weren't monitoring properly, they missed how the borrower's quality was getting worse (from $p$ to $p'$). According to the RNND model, the bank should make extra returns ($R_2$) to make up for their initial costs ($\gamma$), but only if they're keeping an eye on things. Instead, Julius Baer kept lending based on old assumptions and collateral that was losing value.

To sum up, Julius Baer \textbf{looked like a relationship lender but didn't really act like one}. They spent money on initial screening and kept the relationship going, but they weren't monitoring or enforcing rules properly. This meant they didn't get the protection from being able to liquidate, and they didn't make the extra money from having special information - basically, they \textbf{didn't fully follow through on relationship banking}.
