This case highlights the core trade-off in relationship banking: banks may offer favorable terms upfront to attract clients, but doing so can endanger long-term financial stability if borrower risk evolves. In the two-period RNND model, banks incur a sunk cost $\gamma$ and demands a discounted initial repayments ($R_1$). The cost is only recouped if the relationship survives into period two, where higher repayments $R_2$ restore the bank's margin:
\begin{equation*}
    pR_1 + p^2R_2 = 1 + \gamma + p
\end{equation*}
However, if borrower risk deteriorates from $p \to p'$ and the bank does not monitor, it misprices $R_2$, and expected repayments fall short. Illiquid collateral worsens the situation---if it collapses in value, the loan becomes effectively unsecured.

Julius Baer's bridge loans to Signa demonstrate this trade-off. To win the client, Baer likely accepted a low $R_1$ or even discounted $R_1$: generous terms, minimal collateral enforcement, and a reliance on soft information. When Signa's risk worsened and collateral lost value, the bank was locked into the relationship and unable to recover its initial concessions. What began as a short-term acquisition strategy ultimately compromised long-term stability, as the repayment structure no longer covered the embedded risk.
