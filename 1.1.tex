\subsubsection*{Two-Period Setup}
We model Julius Baer's lending relationship with Signa as a two-period lending game in a RNND environment:
\begin{itemize}
    \item \textbf{Period 0}: Julius Baer incurs sunk information cost $\gamma$ to underwrite and understand Signa's risk.
    \item \textbf{Period 1}: It lends 1 unit (bridge loan).
    \item \textbf{If Signa succeeds (with probability $p$)}: It receives follow-up funding in Period 2 (new loan of 1 unit), repaid with $R_2$.
    \item The total expected bank cost is $1 + \gamma + p$.
    \item The total expected repayment is $pR_1 + p^2R_2$.
\end{itemize}
Under competitive zero-profit equilibrium, we have:
\begin{equation}
    pR_1 + p^2R_2 = 1 + \gamma + p \tag{3.23 of coursebook}
\end{equation}

\subsubsection*{Outside Bank Threat at $t = 1$}
If Signa had tried to switch to another lender at $t = 1$, the new bank would have needed to incur the information cost $\gamma$ for one period of lending, and thus:
\begin{equation}
    R_2 = \frac{1 + \gamma}{p} \tag{3.24 of coursebook}
\end{equation}
Combining these two conditions gives the equilibrium level of $R_1$:
\begin{equation}
    R_1 = \frac{1 + \gamma \times(1-p)}{p} < R_2 = \frac{1 + \gamma}{p}
\end{equation}
This creates a \textbf{lock-in effect}, where Julius Baer can extract profits from the ongoing relationship.

\subsubsection*{Incentives}
\begin{itemize}
    \item \textbf{Julius Baer} expected to earn future income from follow-on lending, repaid via $R_2$, without needing to re-invest in due diligence. As a result, the bank is willing to take a loss in the first period. The coursebook has extensive explanations on this part.
\end{itemize}

\subsubsection*{Risks}
\begin{itemize}
    \item Misaligned incentives and monitoring gaps: In a two-period relationship, low initial repayments and the need to preserve future profits can discourage banks from funding positive-NPV projects.
    \item Lock-in and excessive risk tolerance: The bank's ex-post monopoly power may lead to leniency in lending standards and an overcommitment to troubled clients. Long-term dependence can result in inefficient capital allocation and difficulty exiting relationships when risks materialize---turning relational strength into a liability.
\end{itemize}