At a private bank like Julius Baer, \textbf{client relationship managers} are often rewarded based on how much new business they bring in, such as winning big clients or arranging large loans. To attract a major client like the Signa Group, they may offer \textbf{customized financing deals}, sometimes with fewer covenants, lower rates, or more flexible terms. These managers usually receive bonuses or recognition once a deal is signed, but there is often \textbf{no penalty later on} if the loan goes bad. This creates a kind of \textbf{``call option'' incentive}, where they benefit from the upside but are not directly affected by the downside risks.

This structure can \textbf{conflict with the bank's central risk management approach}, which is supposed to:
\begin{itemize}
    \item Check credit quality \textbf{independently of sales pressure};
    \item Set strict standards for things like collateral and loan terms;
    \item Make sure that people who take risks are also responsible if those risks turn out badly.
\end{itemize}

In the Signa case, relationship managers at Julius Baer were focused on building the relationship and winning the client. They helped arrange \textbf{complex loans backed by illiquid equity} and participated in \textbf{related-party transactions} that were hard to evaluate and monitor. These kinds of structured products may have helped close the deal, but they \textbf{did not fit well into the bank's standard risk controls}.

The \textbf{CHF 144 million write-down} shows that the risks in these deals were not properly included in the bank's central oversight. While the managers got credit for bringing in Signa, the losses were absorbed by the bank as a whole. In the end, the strategy led to \textbf{regulatory criticism and investor concern}, and revealed a big problem: \textbf{the incentives of individual managers were not aligned with the bank's long-term risk management goals}.

To address this misalignment, banks need to implement \textbf{longer-term incentive structures} that consider loan performance over time. This could include:
 Deferred compensation that vests over the life of the loans;
Clawback provisions for bonuses if loans go bad;
Performance metrics that balance new business with portfolio quality.

Additionally, the central risk management function needs \textbf{stronger independence and authority} to review and approve complex transactions before they're finalized;
monitor ongoing exposure to high-risk clients;
enforce consistent standards across all relationship managers.

The Signa case demonstrates that without these safeguards, the pursuit of short-term gains by the client manager can lead to significant long-term losses for the bank. The bank's reputation and financial stability ultimately depend on maintaining proper risk controls, even when competing for prestigious clients.
